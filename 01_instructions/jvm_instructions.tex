\documentclass[xetex, 20pt]{beamer}
\usepackage{amsmath, amsfonts, epsfig, xspace}
\usepackage{pstricks,pst-node}
%\usepackage[utf8]{inputenc}
\usepackage[normal,tight,center]{subfigure}
\setlength{\subfigcapskip}{-.5em}
\usepackage{beamerthemesplit}
\usepackage{listings}
\usepackage{xcolor}
\usepackage{graphicx}
\usepackage{tikz}
\usepackage{textpos}
\usetikzlibrary{patterns}
\usetheme{libjava-theme}
\newcommand{\putat}[3]{\begin{picture}(0,0)(0,0)\put(#1,#2){#3}\end{picture}} % just a shorthand
\newcommand<>{\fullsizegraphic}[3]{
  \begin{textblock*}{0cm}(#1,#2)
  \includegraphics[width=\paperwidth]{#3}
  \end{textblock*}
}
\date{}
\author{marcel.hlopko@fit.cvut.cz}
\title{Java Instruction Set}

\begin{document}

\maketitle

\begin{frame}
	\frametitle{Java Bytecode}
	Just a byte array, stream of instruction codes together with their arguments
\end{frame}

\begin{frame}
	\frametitle{Java Instructions}
	mnemonics [<param1> <param2> ...]
	Instructions manipulate the program stack
\end{frame}

\begin{frame}
	\frametitle{Conventions}
	i	integer

	l	long

	s	short

	b	byte

	c	character

	f	float

	d	double

	a	reference
\end{frame}

\begin{frame}
	\frametitle{xconst <const>}
	put constant onto the stack

	e.g. iconst 200

	shorthands: iconst\_1 ...
\end{frame}

\begin{frame}
	\frametitle{xload <index>}
	load x from local variable <index> onto the stack
\end{frame}

\begin{frame}
	\frametitle{xstore <index>}
	store x from the stack into local variable <index>
\end{frame}

\begin{frame}
	\frametitle{getfield <index>}
	takes objectRef from the stack and pushes its field val. Field is specified by fieldRef at CP at <index>
\end{frame}

\begin{frame}
	\frametitle{putfield <index>}
	pops objectRef and new value and stores it into its <index> field

	putstatic, getstatic
\end{frame}

\begin{frame}
	\frametitle{athrow}
	pops exception instance and throws it
\end{frame}

\begin{frame}
	\frametitle{return, xreturn}
	returns from the method (xreturn with TOS)
\end{frame}

\begin{frame}
	\frametitle{ifnull <offset>}
	pops value, if null, then jump relatively by <offset>
\end{frame}

\begin{frame}
	\frametitle{jumps}
	ifnull, ifnonnull, ifge, ifgt, ifeq, ifne, iflt, ifle, if\_cmple ...
\end{frame}

\begin{frame}
	\frametitle{Math}
	xadd, xsub, xdiv ...
\end{frame}

\begin{frame}
	\frametitle{Conversions}
	i2b, i2c, i2d, ...
\end{frame}

\begin{frame}
	\frametitle{checkcast <index>}
	checks TOS if its castable to classRef in CP at <index> 
\end{frame}

\begin{frame}
	\frametitle{ldc... <index>}
	push constant from CP
\end{frame}

\begin{frame}
	\frametitle{monitorenter, monitorexit}
	enters/exits monitor :)
\end{frame}

\begin{frame}
	\frametitle{invokevirtual, invokestatic, invokeinterface, invokespecial}
	generally takes methodRef from CP at <index> and invokes it.
\end{frame}

\begin{frame}
	\frametitle{and few more}
	JVM spec is available online, go and read :)
\end{frame}

\begin{frame}
	\frametitle{Questions and Discussion}
\end{frame}

\end{document}

