\documentclass[xetex, 20pt]{beamer}
\usepackage{amsmath, amsfonts, epsfig, xspace}
\usepackage{pstricks,pst-node}
%\usepackage[utf8]{inputenc}
\usepackage[normal,tight,center]{subfigure}
\setlength{\subfigcapskip}{-.5em}
\usepackage{beamerthemesplit}
\usepackage{listings}
\usepackage{xcolor}
\usepackage{graphicx}
\usepackage{tikz}
\usepackage{textpos}
\usetikzlibrary{patterns}
\usetheme{libjava-theme}
\newcommand{\putat}[3]{\begin{picture}(0,0)(0,0)\put(#1,#2){#3}\end{picture}} % just a shorthand
\newcommand<>{\fullsizegraphic}[3]{
  \begin{textblock*}{0cm}(#1,#2)
  \includegraphics[width=\paperwidth]{#3}
  \end{textblock*}
}
\date{}
\author{marcel.hlopko@fit.cvut.cz}
\title{Classfile Format}

\begin{document}

\maketitle

\begin{frame}
	\frametitle{Java Classfile}
	Does not contain bytecode only! Bytecode is only a small part of classfile
\end{frame}

\begin{frame}
	\frametitle{Magic Number}
	4 bytes

	0xCAFEBABE
\end{frame}

\begin{frame}
	\frametitle{Version}
	2x2 bytes

	jdk 5 -> 0, 49

	jdk 6 -> 0, 50

	jdk 7 -> 0, 51
\end{frame}

\begin{frame}
	\frametitle{Constant Pool}
	2 bytes for cp\_size

	cp\_info * (cp\_size - 1) bytes of constant pool
\end{frame}

\begin{frame}
	\frametitle{Access Flags}
	2 bytes

	public, final, interface, abstract, super
\end{frame}

\begin{frame}
	\frametitle{this class}
	2 bytes index into CP (ClassInfo)
\end{frame}

\begin{frame}
	\frametitle{super class}
	2 bytes index into CP (ClassInfo)
\end{frame}

\begin{frame}
	\frametitle{Interfaces}
	2 bytes ifaces\_count
	
	2 x ifaces\_count bytes of indices into CP (ClassInfos)
\end{frame}

\begin{frame}
	\frametitle{Fields}
	2 bytes fields\_count

	field\_info * fields\_count bytes
\end{frame}

\begin{frame}
	\frametitle{Methods}
	2 bytes methods\_count

	method\_info * methods\_count bytes
\end{frame}

\begin{frame}
	\frametitle{Attributes}
	2 bytes attributes\_count

	attribute\_info * attributes\_count bytes
\end{frame}

\begin{frame}
	\frametitle{Constant Pool Content}

	Literal values (integers, utf8 strings ...)

	References
\end{frame}

\begin{frame}
	\frametitle{Example: ClassRef}

	Struct containting an index into CP where utf8 string describing FQDN of class is stored
\end{frame}

\begin{frame}
	\frametitle{Example: MethodRef}

	Struct containting 2 indices:
	\begin{itemize}
		\item one for ClassRef
		\item one for NameAndTypeInfo
	\end{itemize}
\end{frame}

\begin{frame}
	\frametitle{Example: FieldRef}

	Struct containting 2 indices:
	\begin{itemize}
		\item one for ClassRef
		\item one for NameAndTypeInfo
	\end{itemize}
\end{frame}

\begin{frame}
	\frametitle{Class Info}
	just index into CP where FQDN name is stored
\end{frame}

\begin{frame}
	\frametitle{Method Info}
	2 bytes access flags

	2 bytes name index

	2 bytes descriptor index

	2 bytes atributes\_count

	attributes\_count * attribute\_info bytes of attribute\_info

\end{frame}

\begin{frame}
	\frametitle{Attribute Info}
	describing various attributes, e.g.
	\begin{itemize}
		\item Constant Value
		\item Exceptions
		\item Inner Classes
		\item Code ...
	\end{itemize}
\end{frame}

\begin{frame}
	\frametitle{Code Attribute}
	max stack

	max locals

	bytecode

	exceptions table
\end{frame}

\begin{frame}
	\frametitle{Questions and Discussion}
\end{frame}

\end{document}

