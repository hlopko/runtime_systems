\documentclass[xetex, 20pt]{beamer}
\usepackage{amsmath, amsfonts, epsfig, xspace}
\usepackage{pstricks,pst-node}
%\usepackage[utf8]{inputenc}
\usepackage[normal,tight,center]{subfigure}
\setlength{\subfigcapskip}{-.5em}
\usepackage{beamerthemesplit}
\usepackage{listings}
\usepackage{xcolor}
\usepackage{graphicx}
\usepackage{tikz}
\usetikzlibrary{patterns}
\usetheme{libjava-theme}
\newcommand{\putat}[3]{\begin{picture}(0,0)(0,0)\put(#1,#2){#3}\end{picture}} % just a shorthand
\newcommand{\yesitem}{\item[\checkmark]}
\newcommand{\noitem}{\item[\texttimes]}
\let\oldfootnotesize\footnotesize
\renewcommand*{\footnotesize}{\oldfootnotesize\tiny}
\newcommand<>{\fullsizegraphic}[3]{
  \begin{textblock*}{0cm}(#1,#2)
  \includegraphics[width=\paperwidth]{#3}
  \end{textblock*}
}
\date{}
\author{marcel.hlopko@fit.cvut.cz}
\title{Garbage Collection}

\begin{document}

\maketitle
%\begin{frame}
%\section{foo}
%\end{frame}
\begin{frame}
	\frametitle{Problem}

	We need to allcoate memory, and we have to free it back.
\end{frame}


\begin{frame}
	\frametitle{Manual Deallocation}
	\begin{itemize}
			\noitem hinders design and extensibility
			\noitem forces bookkeeping
			\noitem GC can be faster\footnote{
				Andrew W. Appel. Garbage collection can be faster than stack
			allocation.  \textit{Information Processing Letters, 25(4):275-279,
		June 1987.}},
		usually comparative\footnote{ 
			Benjamin Zorn. The measured cost of conservative garbage
		collection.  \textit{Software --- Practice and Experience},
		23(7):733-756, 1993}
	\end{itemize}
\end{frame}

\begin{frame}
	\frametitle{Manual Deallocation}
	A lot of large applications ended up implementing some form of automatic
	memory management themselves.
\end{frame}

\begin{frame}
	\frametitle{Let's Make a GC}
	\begin{itemize}
		\item How to find live / gargabe objects\qmark
		\item How to reclaim garbage\qmark
	\end{itemize}
\end{frame}

\begin{frame}
	\frametitle{Object liveness}
	Root Set \ampersandmark Reachability
\end{frame}

\begin{frame}
	\frametitle{Finding Live Objects}
	\begin{itemize}
		\item Reference Counting
		\item Tracing
	\end{itemize}
\end{frame}

\begin{frame}
	\frametitle{Reference Counting}
	\begin{itemize}
		\item \emphtext{refCount} stored in object header
		\item object reclaimed when \emphtext{refCount} reaches 0
	\end{itemize}
\end{frame}


\begin{frame}
	\frametitle{Reference Counting}
	\begin{itemize}
		\yesitem incremental nature
		\yesitem easy to make real-time
		\yesitem degrades well with full heap			
		\noitem \emphtext{cycles}
		\noitem overhead
		\noitem fragmentation
	\end{itemize}
\end{frame}

\begin{frame}
	\frametitle{Can we do better?}

	Tracing algorithms
	\begin{itemize}
		\item GC is not running always
		\item GC scans the whole heap
	\end{itemize}
\end{frame}

\begin{frame}
	\frametitle{Mark-Sweep}
	\begin{itemize}
		\item marking live objects
		\item sweeping all unmarked
	\end{itemize}
\end{frame}

\begin{frame}
	\frametitle{Mark-Sweep}
	\begin{itemize}
		\yesitem handles cycles
		\noitem fragmentation
		\noitem bigger heap - longer run
	\end{itemize}
\end{frame}

\begin{frame}
	\frametitle{Mark-Compact}
	\mottotext{How about "defragmenting" living objects?}
	\begin{itemize}
		\yesitem solves fragmentation
		\yesitem solves referential locality
		\yesitem simple allocation
		\noitem still multiple passes over heap
	\end{itemize}
	
\end{frame}

\begin{frame}
	\frametitle{Baker}
	\mottotext{How about copying objects on the fly?}
	2 semispaces, only one used at the time
\end{frame}

\begin{frame}
	\frametitle{Baker}	
	\begin{itemize}
		\yesitem only one run over the heap
		\noitem memory demanding
		\noitem still has to stop the world
	\end{itemize}
\end{frame}

\begin{frame}
	\frametitle{Incremental Collectors}
	GC runs in parallel with application (\emphtext{mutator})

	\emphtext{relaxed consistency}

	(conservative approximation of the true reachability graph)
\end{frame}

\begin{frame}
	\frametitle{Tricolor Marking}
	\emphtext{black} will be retained

	\emphtext{white} will be collected

	\emphtext{grey} will be expanded
\end{frame}

\begin{frame}
	\frametitle{Incremental Algorithm}
	\begin{itemize}
		\item color root set grey
		\item while grey set is not empty
			\begin{enumerate}
				\item take grey object
				\item color all his white children grey
				\item color object black
			\end{enumerate}
	\end{itemize}
\end{frame}

\begin{frame}
	\frametitle{Mutator mutates}

	Mutator can assign a \emphtext{white} child into the \emphtext{black} object

	Solutions:
	\begin{itemize}
		\item read barrier
		\item write barrier
	\end{itemize}
\end{frame}

\begin{frame}
	\frametitle{Read Barrier}
	Detect reads of white objects and color them grey immediately.
\end{frame}

\begin{frame}
	\frametitle{Read Barrier}
	\begin{itemize}
		\yesitem mutator "keeps itself" inside grey wave
		\noitem usually too expensive
	\end{itemize}
\end{frame}

\begin{frame}
	\frametitle{Baker's Read Barrier GC}
	\begin{itemize}
		\item \emphtext{atomic} flip 
		\item mutator resumed
		\item fromspace access is \emphtext{trapped}
		\item fromspace object copied first
		\item copied objects colored grey
	\end{itemize}
\end{frame}

\begin{frame}
	\frametitle{Write Barrier}
	Detect writes into black objects.
\end{frame}

\begin{frame}
	\frametitle{Snapshot-at-beginning}
	Does not allow the pointer to be overwritten, it stores the original
	pointer "off to the side".

	Very \emphtext{conservative}
\end{frame}

\begin{frame}
	\frametitle{Incremental Update}

	When a reference to a white object is stored into 
	the black object, the black object can be greyed (Steele) or
	the white object can be greyed (Dijkstra)
\end{frame}

\begin{frame}
	\frametitle{Object Age}
	\begin{center}
		80-98\% short lived objects.

		Many survivors will survive a lot.
	\end{center}
\end{frame}

\begin{frame}
	\frametitle{Generations}
	Heap will be divided into multiple sections, each containing different generations of objects.
\end{frame}

\begin{frame}
	\frametitle{Generations}
	\begin{itemize}
		\item \emphtext{young} generation, GC'ed very often
			(JVM default size: 640kb)

		\item \emphtext{old} generation, GC'ed less often
			(JVM default size: 64mb)
	\end{itemize}
\end{frame}

\begin{frame}
	\frametitle{Remembered Sets}

	References from old gen. to eden are recorded.

	All such old gen. objects are part of the root set for young gen.
\end{frame}

\begin{frame}
	\frametitle{Questions And Discussion}

	\emphtext{Literature:}
	λ
	Uniprocessor Garbage Collection Techniques, Paul R. Wilson
\end{frame}
\end{document}

